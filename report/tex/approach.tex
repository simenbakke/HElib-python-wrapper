\chapter{Approach}

This chapter outlines the methodology and actions which will be taken to create a Fully Homomorphic Encryption API in Python. We will also take a look at how we can test the Python API to measure the difference in performance between the C++ library and the Python library. We will explain the approach of choosing a Python wrapper for the C++ code. And we will look at how we can determine possible usages and limitations of the library.





\section{Objectives}

The main objective of this thesis is to create a Fully Homomorphic Encryption API in Python and measure the performance to get a sense of how far we have come in the development and implementation of FHE schemes. The Python API will be based on an already existing FHE library. The Python API should support a multitude of functions in order to classify it as a FHE API. Ideally the Python API should not be significantly slower than the original library in terms of compilation time and runtime. Lastly we want to highlight the benefits/drawbacks of having a FHE scheme in Python and find potential usages of the API.
\\\\
The thesis work is structured into the following three stages
\\\\
1. A design stage\\\\
2. An implementation stage\\\\
3. Measurement, analysis and comparison stage

\section{Design Stage}

In the design stage we will discuss the necessary steps to design a solution which satisfies the objectives mentioned above. For simplicity sake the design stage is divided into several steps needed to fulfill the requirements of the objectives, these are outlined below.
\\\\

1. Research and choose a suiting C++ wrapper for python\\\\
2. Create an API which is easily understood\\\\
3. Find a suitable model for testing the API

\subsection{C++ wrapper for Python}
In this section we will examine different C++ wrappers to find one that is suitable to wrap the FHE scheme to python. Before choosing the wrapper the FHE scheme implementation should be examined to see if there are some special functionality the wrapper should support. If the FHE scheme have a lot of user defined objects or any other functionality, the wrapper chosen should be able to handle this in a somewhat simple matter. The main goal of this exercise is to find a wrapper to create a Python API from an already existing FHE scheme. The wrapper should ideally have a good performance i terms of runtime and the API created with the wrapper should be well structured and easily organizable.

\subsection{Create the API}
There are currently few options when it comes to FHE libraries in Python. When the Python API is created it is important that the API is well structured, well documented and it should include examples for potential usages. Ideally the Python API should be as similar to the underlying library as possible, making it easier for the user to find valuable documentation on how to use the library. The library being wrapped to Python is most likely an extensive library which means the user should be able to use the documentation for the API combined with the underlying library documentation for a more thorough understanding on how to use the API to the full extent. Any differences between the API and the underlying library should be highlighted making it easy for the user to see the parallels between the API and the library. As the underlying library is most likely an extensive library it will consist of many files. When creating the Python API this needs to be taken into consideration as the API should be presented in a lucid manner corresponding to the underlying library. As the underlying library will most likely be large it should also be taken into consideration how much of the library should be wrapped to Python to support the functionality required by an FHE scheme.

\subsection{Find a suitable model for testing the API}
Once the API is created the performance of the Python API should be measured with regards to the existing library in C++. A benchmark test should be created on the existing C++ library which can be used in comparison to the Python API. The tests created should include most of the functions wrapped to Python to see if there are some operations or functions which are slower than others in regards to the benchmarks. If a function or an operation is slower than others it should be investigated to provide an explanation why this is the case and possible solutions to the result. The tests implemented should not only serve as a measurement in performance but also as examples to potential usages of the library. The tests should show how the library can be used in different scenarios and also show the limitations to the library. It's important that the tests highlight the benefits and drawbacks of having the FHE scheme implemented in Python compared to the C++ implementation. If there are other implementations of a FHE scheme in Python this could be used in the testing phase to measure the difference in performance in terms of runtime and functionality compared to the API implemented.

During the research stage of the assignment an implementation of the FHE scheme in Python was found. This implementation does have some limited functionality compared to the existing C++ library, but it can be used as a comparison to our implementation. We should look at this library to compare the difference in performance and functionality to our implementation of the HElib library.

\newpage
\section{Implementation stage}
The implementation stage is divided into a few objectives to satisfy the goals outlined in the design stage.


1. Choose an underlying FHE scheme implementation\\\\
2. Wrap the code to Python\\\\
3. Create a structured Python API \\\\
3. Create tests to measure the performance \\\\

\subsection{Choose a Fully Homomorphic Scheme implementation}
The first task of the implementation stage is to choose a suiting FHE scheme which will serve as the underlying library of the Python API. The FHE scheme chosen should be up-to date and therefore be based on one of the 2nd generation of homomorphic cryptosystems. These includes The Brakerski-Gentry-Vaikuntanathan cryptosystem (BGV) \cite{cryptoeprint:2011:277}, Brakerski's scale-invariant cryptosystem \cite{cryptoeprint:2012:078}, The NTRU-based cryptosystem due to Lopez-Alt, Tromer, and Vaikuntanathan (LTV)\cite{cryptoeprint:2013:094} and The Gentry-Sahai-Waters cryptosystem (GSW) \cite{cryptoeprint:2013:340}.

Of the 2nd generation homomorphic cryptosystems there are three open source libraries. The HElib library \cite{halevi2014helib} that implements the BGV cryptosystem with the GHS optimization, The FHEW library \cite{ducas2015fhew} which implements a combination of Regev's LWE cryptosystem \cite{regev2009lattices} with the bootstrapping techniques of Alperin-Sheriff and Peikert \cite{cryptoeprint:2014:094} and the TFHE library\cite{chillotti2016tfhe} which proposes a faster variant over the Torus. Of the three libraries the HElib library was chosen as a basis for the Python API. HElib offers some interesting features like ciphertext packing and recryption.
\newpage

\subsection{Wrap the code to Python}
The HElib library is chosen as the underlying library for the Python API. Before wrapping the code to Python and create the Python API a suiting wrapper should be chosen. The HElib library uses the NTL mathematical library which means it most likely will use user defined objects and types. The wrapper chosen should be able to handle this in a good manner. The main goal when exporting the library to Python is to create a shared library which can be imported as a module in Python. This module should be directly and easily accessible from Python and it should be complete in the sense that the user should be able to use all the functionality in the module without the need specify return values or create any form of conversions of C++ types.\\\\
In the Python module of the HElib library the main functionality of the library will be exported, this means as the library is quite extensive, that there will still be parts of the library which will not be exported. As the module might be extended in the future to support more functionality it should be structured in such a way to make it easy for anyone not familiar with the project to understand how it works.  Also the HElib library uses a technique called ciphertext packing to create a bulk of data to which an instruction is performed, called SIMD(Single Instruction Multiple Data). These ciphertexts are vectors. This is something to keep in mind when choosing the wrapper as vectors are not a familiar concept in Python. \\\\
Another thing to keep in mind, the HElib library consists of multiple functions which have the same function name, so called overloaded functions. These functions have the same name, but with different input types. When the function is called the interpreter recognized the input types and makes a function call to the correct function. This is also a concept which is not familiar in Python. Before choosing the wrapper we should consider the following requirements for which the wrapper should handle.

\textbullet C++ Structures\\\\
\textbullet User defined objects/types\\\\
\textbullet Vectors\\\\
\textbullet Overloaded functions\\\\
\textbullet Pointers\\

The wrapper should not only support the requirements mentioned above. Ideally the wrapper chosen should fast, in terms of compilation and runtime. It should be simple Among the more known wrappers from C++ to Python we have Boost, Ctypes, Swig and Python C-extensions.

\subsubsection{C-extenstions}
Python C-extensions is a feature which is implemented within Python and does not require any extra software. It is a good software if the objective is to wrap only small parts of code. The main problem with Python C-extension is that it is low level, making it time consuming to expose functions over to Python. Other drawback of the software is that it is prone to memory leaks and is Python version dependent. As the HElib library consists of a rather large amount of code which is going to be exposed to Python, this is not an option.

\subsubsection{SWIG}
On the other side of the spectrum we have SWIG. SWIG stands for Simplified Wrapper and Interface Generator and is a software designed to automate the process of wrapping code to Python seamlessly. Automatically wrap the code from C++ to Python would be ideal if it can handle all the requirements mentioned above. The problem is, it doesn't. Although it does handle standard vectors and pointers fairly well, it only have partial support for overloaded functions and custom vectors have to be wrapped manually. Another key point about SWIG is that, if you do not wish to wrap a whole file one must specify the desired functions to be wrapped. The HElib library is quite large and the majority of the code does not have to be wrapped to Python to be able to support the desired functionality. This means by using SWIG one have to specify exactly which functions to be wrapper combined with the manual wrapping required the user defined objects makes the automation process obsolete.

\newpage
\subsubsection{Ctypes}
So, the C-extensions are too low level making the code wrapping slow and tedious. SWIG automates this process, but since the HElib consists of many overloaded functions and user defined objects there will not be much code left to be automatically wrapped to Python. Another option is Ctypes. Ctypes is a standard library module in Python, which adds the standard C types to python. At runtime it can load a shared library and import it's symbols, allowing a Python application to make functions calls into the library without any special preparations. It offers extensive facilities to create, access and manipulate simple and complicated C data types in Python. \\\\
With Ctypes the HElib library can be compiled as a shared library and then imported into Python. The problem with this approach is that one still have to specify the return value of the different functions. For any special objects like a vector or overloaded functions which is not a standard in Python, a converter would need to be written in C++ to be exported to Python. This would result in a combination of C++ and Python code to make the library being useful, which would make the API unstructured and messy. This also defeats the purpose of the goal which is to create a shared library which can be imported as a Python module and be directly accessible in a simple matter.

\subsubsection{Boost}
The last wrapper considered is Boost Python. The main goal with Boost is to enable the user to export C++ functions and classes to Python using nothing more than a C++ compiler \cite{abrahams2003building}. Boost is a C++ library where the syntax is similar to C++ and it's designed to be minimally intrusive to the existing C++ code. Having the syntax similar to C++ is beneficial at it does not require the user to learn an additional language as it would with other wrappers like SWIG or SIP. With Boost the HElib library can be exported to python with rewriting the original code, this is a desired feature, as the library is complex making it hard to rewrite. Boost also supports structures, vectors, pointers, user defined objects and overloaded functions.

\newpage

\subsection{Create a structured Python API}
This might sound like it should not be a big deal, but using a wrapper which allows one to export functions while maintaining them in a structured manner is important, especially when the project grows large. A large project will consist of multiple files, many functions and classes. Exposing the project code without a proper structure does not only make the work for the one exporting the code more difficult, but even more so for the ones who are not familiar with the project. When dealing with a small project one could gather all the exported code within a single file and it would be no problem. But when dealing with a large project which consists of multiple files, this would be very unorganized and hard for anyone to see how the code is structured. \\\\
Ideally the structure of the exported code should be similar to the structure of the original code making it easy to see what file it corresponds to in the original code. For example when dealing with overloaded functions which is functions having the same name but with different input parameters. If an overloaded function is spread over multiple files and the exported code is gathered in a single file, which means all the overloaded functions are gathered in the same file. This would make it very hard to see what context each of the overloaded function relates to. Now, you are probably asking yourself, why does this matter as long as all the functions are exported? For starters, the concept of overloaded functions are not relatable to Python. When a C++ compile the code it uses a technique called name mangling which encode functions and variable names into unique names so that linkers can separate common names in the language making it possible to user overloaded functions, namespaces and templates. \\\\
Although Boost Python do have support for exporting overloaded functions it means that the functions exported will have to be given a unique name. One could organize the different function by giving them a unique name which corresponds to which context the function are being used, but this will result in rather large function names. By having the functions separated in files which corresponds to the original code, it is easy to see where and in what context the functions belongs to. Another important issue is when the original library is undergoing changes and updates. During the change or the update only parts of the code will be updated. If there is no structure to the exported library it would be hard to find out what part of the exported code which needs to be changed.

\newpage

\subsection{Create tests to measure the performance}
Once the code has been exported and the projects is organized in a structured manner it is time to test the exported library functionality. With the tests we want to see the difference in performance between the original library and the exported module. There will be created a test which involves the majority of the functions with a set of fixed parameters which will run in a loop where the goal of the test is to measure the difference in performance when multiple functions are called multiple times. The amount of rounds in the loop will be the variable to see if there are more/less difference in performance depending on how many times an exported function is called. A timer will be used to measure the runtime of the different functions to see if there are some functions that are slower than others. The runtime of the encryption/decryption and total runtime will also be measured.\\\\
The second test will compare the implementation of the exported library with another project by Ibarrondo \cite{Pyfhelhelib}. In the project an abstraction of HElib (Afhel) is created in C++ which includes the most important HElib operations. This abstraction is wrapped using Cython, creating what he calles Python for HElib (Pyfel). The goal of the test is to measure the difference in performance and functionality between the implementations.\\\\
In the last test and implementation of a server and a client will be implemented. This should simulate a client and a cloud server. The client will encrypt some data, which are sent over to the server. The server will perform some computation and send the data back to the client. The client will then decrypt the data making sure the data is correct. With this implementation we want to measure the difference in performance between different scenarios.\\\\
1. A baseline where the client sends the data in plaintext to the server which performs the computation and send it back.\\\\
2. The encrypted data is stored on the server. The client fetches the data, decrypts it, performs the computation, encrypts it again and stores in on the server.\\\\
3. Use the HE implementation to perform the computation on the server.
