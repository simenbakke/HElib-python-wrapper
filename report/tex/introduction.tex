\chapter{Introduction}

\begin{enumerate}

 When dealing with sensitive data we need to make sure that the data is secure. This means, if the data is leaked due to a security breach, error in the system or any other reasons, we need to make sure the data are unrecognizable for anyone who is not authorized. This is commonly solved by using an encryption algorithm to encrypt the data, making it unreadable to anyone who do not know the private key used during the encryption. As more and more businesses and organizations are moving away from having a their own dedicated server to renting server space in large server parks, like a cloud, security are becoming a bigger concern than ever before. Traditionally businesses and organizations would have their own private server which would handle computation and storing of the data. The only people having direct access to the server would be authorized personnel hired by the business or the organization. Meaning the only real threat from a potential attacker would come from outside the business/organization, as any attacks from inside the business/organization would be met with prosecution and lawsuit.
\\\\
 As the trend of cloud services have exploded in the last years this have changed. We no longer have control over the physical location of the server, and we now have to trust the cloud service to uphold access control agreed upon. When storing data on cloud servers we are faced with not only the same security threats as on the private server, but also additional threats specific for cloud computing, like side channel attacks[12], virtualization vulnerabilities and abuse of cloud services. Cloud administrators or unauthorized users might access the data and obtain sensitive information for various motivations. Because of the increased threat when storing data in a cloud environment the data need to be encrypted at all times with our own encryption key, as even the cloud service cannot be trusted.
\newpage
 If we use the cloud service to only store data, then this is pretty simple, but what if we want to do some computation on the data stored? One way to do this is to encrypt the sensitive data and store this on the cloud. To perform a computation the data would be downloaded to a private server and decrypted. The computation is performed and the data is encrypted and uploaded back to the cloud. The problem with this approach, it's  inefficient, especially when a lot of computation is done and it also requires a private server to perform the computation. Another approach is to use a Homomorphic Encryption(HE) scheme which enables us to perform computation on encrypted data without decrypting it first. This means, we can perform computation on ciphertexts generating an encrypted result which, when decrypted, matches the result of operations performed on the plaintext. By using Homomorphic Encryption we can perform the computation without ever exposing the data which means we can ensure that the data is secure at all times during the computation.
\\\\
Why do we want Homomorphic Encryption? As shown in the previous example it allows us to do computation on our data without ever exposing the data during the computation. This essentially means that we can now export the computation work to a third party like a cloud service. We no longer have to pay for expensive servers to perform heavy computations, we can export the job an external server which is much cheaper in term of power consumption, server maintenance and bandwidth cost. This also allows us to pay for exactly the amount of storage and computational power needed, which can dynamically increased when needed. But is also have other benefits like hospital record can be examined without compromising patient privacy and financial data can be analyzed without opening it up to theft.
\\\\
If Homomorphic Encryption is so great why are not everyone using it already? Even though there are many benefits which comes with using Homomorphic Encryption, there are some issues.
First of all, performance. The first Fully Homomorphic Encryption scheme by Craig Gentry[1] was taking 100 trillion times as long to perform computations on encrypted data than plaintext computation. In this paper we will take a look at a FHE library called HElib which is an open source library in C++. With Boost Python we have created a Python API of this library and will use this to see how far the FHE schemes have come and look at practical usages and/or which challenges still remain for FHE schemes.  

\end{enumerate}
